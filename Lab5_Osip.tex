\documentclass[12pt,a4paper]{article}
\usepackage[russian]{babel}
\usepackage{amsmath}

\usepackage{verbatim}
\usepackage{setspace}
\singlespacing
\usepackage{graphicx}
\usepackage{caption}
\usepackage{float}
\usepackage{amsfonts}
\usepackage{amssymb}
\usepackage{graphicx}
\usepackage{enumitem}
\usepackage{xcolor}
\usepackage{subcaption}
\usepackage{pdfpages}

\newtheorem{theorem}{Theorem}

\begin{document}
	\begin{description}[font=$\bullet$~\normalfont\scshape\color{red!50!black}]
		\item[Метод Ньютона.Кратные корни.] Покажем, что в окрестности корня кратности два скорость сходимости метода - линейная. \\
		Пусть решается уравнение \begin{equation}
			f(x) = 0
		\end{equation} кратный корень $x^{*}$ которого локализован на $[a,b]$.
		Итерационный метод Ньютона описывается формулой $x^{n+1} = \varphi(x^n)$, где $\varphi(x) = x - \frac{f(x)}{f'(x)} $. \\
		Предположим, что $\exists \lim\limits_{n\to\infty}x^n = c$. Тогда если $\varphi(x)$ непрерывна в точке $c$, то  она является решением уравнения \begin{equation}
			\varphi(x) = x
		\end{equation}
		 (Именно для этого уравнения приводят к подобному виду, удобному для итераций, и строят итерационную последовательность именно в таком виде). Если же разрывна, то можно лишь утверждать, что $\lim\limits_{n\to\infty}\varphi(x^n) = c$. Заметим, что разрыв будет только тогда, когда $f'(c) = 0$. \\
		 Вопрос: почему корень уравнения (1) является корнем уравнения (2)? Иными словами, на чем базируется Ньютона, сводя задачу (1) к (2)?
		 Рассмотрим предел
		\begin{equation}
					 	\lim\limits_{x\to x^*}\varphi(x) = \lim\limits_{x\to x^*}(x - \frac{f(x)}{f'(x)}) = x^* - \lim\limits_{x\to x^*}\frac{f(x)}{f'(x)}
		\end{equation}
		 Если корень был простой, то 
		 $$
		  \lim\limits_{x\to x^*}\frac{f(x)}{f'(x)} = \frac{1}{f'(x^*)} \lim\limits_{x\to x^*}f(x) = 0
		  $$
		  Если же, как и предполагалось выше, корень был кратности два, то по правилу Лопиталя
		  $$
		  	\lim\limits_{x\to x^*}\frac{f(x)}{f'(x)} = \lim\limits_{x\to x^*}\frac{f'(x)}{f''(x)} = \frac{1}{f''(x^*)} \lim\limits_{x\to x^*}f'(x) = 0
		  $$
		  поскольку корень - кратности два (значит $f''(x^*) = 0$) и вторая производная непрерывна
		  Таким образом мы доказали, что хоть $\varphi(x)$ и разрывна, но разрыв этот первого рода и устраним (и можно было бы вообще доопределить по непрерывности значением предела  $x^*$). \\
		 Производная $\varphi$
		  $$
		  	\varphi'(x) = \frac{f(x) f''(x)}{f'^2(x)}, x \neq x^*
		  $$
		\begin{equation}
			\lim\limits_{x\to x^*}\varphi'(x) = f''(x*) \lim\limits_{x\to x^*}\frac{f(x)}{f'^2(x)} 
		\end{equation}
		Вновь по правилу Лопиталя
		$$
			\lim\limits_{x\to x^*}\frac{f(x)}{f'^2(x)} = \lim\limits_{x\to x^*}\frac{f'(x)}{2f'(x)f''(x)} = \frac{1}{2f''(x^*)}
		$$
		Таким образом, $\lim\limits_{x\to x^*}\varphi'(x)$ = 1/2. Следовательно, по свойствам пределов, $|\varphi(x)| < 1$ в некоторой окрестности $x^*$. Это влечет сходимость итерационного метода, а также то, что порядок сходимости будет по крайней мере первый. \\
		Предположим, что порядок сходимости метода - второй, то есть
		\begin{equation}
			|x^{n+1} - x^*| \leq A |x^n - x^*|^2
		\end{equation}
		Поскольку
		$$
			|x^{n+1} - x^*| = |\varphi(x^n) - \varphi(x^*)| = |\frac{1}{2} (x^n - x^*) + o(x - x^*)|
		$$
		(Разложили по формуле Тейлора).
		Тогда (5) примет вид
		\begin{equation}
			|\frac{1}{2} (x^n - x^*) + o(x - x^*)| \leq A |x^n - x^*|^2
		\end{equation}
		И должно выполняться для любого $n$. Очевидно, что получено противоречие: слева стоит бесконечно малая первого порядка порядка, тогда как $|x - x^*|^2$ - второго. \\
		Таким образом, порядок сходимости - линейный.
	\end{description}
\end{document}